% Created 2021-02-04 Thu 18:50
% Intended LaTeX compiler: pdflatex
\documentclass{article}
\usepackage[utf8]{inputenc}
\usepackage[T1]{fontenc}
\usepackage{graphicx}
\usepackage{longtable}
\usepackage{hyperref}
\usepackage{natbib}
\usepackage{amssymb}
\usepackage{amsmath}
\usepackage{geometry}
\geometry{a4paper,left=2.5cm,top=2cm,right=2.5cm,bottom=2cm,marginparsep=7pt, marginparwidth=.6in}
\usepackage[utf8]{inputenc}
\usepackage[T1]{fontenc}
\usepackage{graphicx}
\usepackage{grffile}
\usepackage{longtable}
\usepackage{wrapfig}
\usepackage{rotating}
\usepackage[normalem]{ulem}
\usepackage{amsmath}
\usepackage{textcomp}
\usepackage{amssymb}
\usepackage{capt-of}
\usepackage{hyperref}
\usepackage{minted}
\author{Curtis D'Alves}
\date{Feb 4th, 2021}
\title{CS 1XA3: Intended Learning Outcomes}
\hypersetup{
 pdfauthor={Curtis D'Alves},
 pdftitle={CS 1XA3: Intended Learning Outcomes},
 pdfkeywords={},
 pdfsubject={},
 pdfcreator={Emacs 27.1 (Org mode 9.4.4)}, 
 pdflang={English}}
\begin{document}

\maketitle
Computer Science 1XA3 is an experiential learning based course designed to teach
students common tools/skills utilized in the practice of software engineering
while learning about underlying theoretical computer science concepts. Students
are expected to already have a beginners knowledge of the python programming
language (previuos or concurrent enrollment in CS 1MD3 is recommended)

\vspace{3mm}
\noindent
CS 1XA3 has the following Intended Learning Outcomes, i.e. 
\begin{itemize}
\item students should have the practical skills to
\begin{itemize}
\item Use Un*x command line interfaces to navigate, create and manipulate filesystems
\item Analyze permissions and modes of Un*x filesystems
\item Apply changes to permissions and modes of Un*x filesystems
\item Use ssh/scp to access remote servers
\item Use the Un*x commands grep/find to locate files/text in large filesystems
\item Use the Un*x commands top*/ps/kill to manage system processes
\item Investigate network connections using the Un*x command netstat
\item Manage a code base using git version control
\item Design and implement a simple webpage using HTML, CSS and Javascript
\item Design and implement a simple web server application using the python Django framework
\item Design and implement a simple SQL database
\end{itemize}

\item and students should have enough working knowledge of computer science concepts to
\begin{itemize}
\item Recognize corresponding tree data structures in file systems
\item Recognize corresponding directed graph data structures in git revision commit history
\item Define regular expressions for string enumeration/matching
\item Recognize basic UI principals used in webpage design
\item Define conceptual models (using UML Diagrams) corresponding to module
relationships in python code
\item Define basic relational algebra equations corresponding to SQL queries
\end{itemize}
\end{itemize}

Learning outcomes for practical skills are largely cummulative, i.e. use of
command line tools will be necessary to use version control which will be
necessary to manage your code to build webpages, which is in turn necessary to
have a functioning server which is also in turn necessary to implement a
database. Therefore evaluation of skills acquired will be reinforced by
evaluation of subsequent skills. Furthermore learning outcomes for computer
science concepts underly the use of corresponding practical skills
\end{document}