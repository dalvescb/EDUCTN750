% Created 2021-04-01 Thu 14:10
% Intended LaTeX compiler: pdflatex
\documentclass[11pt]{article}
\usepackage[utf8]{inputenc}
\usepackage[T1]{fontenc}
\usepackage{graphicx}
\usepackage{grffile}
\usepackage{longtable}
\usepackage{wrapfig}
\usepackage{rotating}
\usepackage[normalem]{ulem}
\usepackage{amsmath}
\usepackage{textcomp}
\usepackage{amssymb}
\usepackage{capt-of}
\usepackage{hyperref}
\usepackage{minted}
\usepackage[dvipsnames]{xcolor}
\usepackage{xcolor-solarized}
\usepackage{sectsty}
\sectionfont{\color{blue}}  % sets colour of sections
\subsectionfont{\color{blue}}  % sets colour of sections
\usepackage{geometry}
\geometry{a4paper,left=2.5cm,top=4cm,right=2.5cm,bottom=4cm,marginparsep=7pt, marginparwidth=.6in}
\author{{\color{blue}Created by: Curtis D'Alves}}
\date{{\color{blue}Due Date: Part 1 Feb 2, Part 2 Feb 16}}
\title{{\color{blue}CS 1XA3 Project 01}}
\hypersetup{
 pdfauthor={{\color{blue}Created by: Curtis D'Alves}},
 pdftitle={{\color{blue}CS 1XA3 Project 01}},
 pdfkeywords={},
 pdfsubject={},
 pdfcreator={Emacs 27.1 (Org mode 9.4.4)}, 
 pdflang={English}}
\begin{document}

\maketitle
\tableofcontents

\newpage

\section{Project Setup}
\label{sec:org19ccbca}
\subsection{Clone Your Repo Into Your Private Directory}
\label{sec:org359d59a}
You must keep your repo inside of your directory located in \texttt{\$HOME/private} on
the \texttt{mac1xa3.ca} server
\begin{itemize}
\item {\color{red}WARNING} not doing so or changing the default permissions
on the private directory will be considered academic dishonesty
\item If you accidentently delete the directory, contact me or one of your TA's (preferably me / dalvescb on discord)
\end{itemize}
\subsection{Create a new folder in your repo}
\label{sec:orgc541743}
On the {\color{purple}master branch}, create a new folder \texttt{CS1XA3/Project01} (where \texttt{CS1XA3} is the already existant root of repo)
\subsection{Add the folowing files}
\label{sec:orgf74326e}
\begin{itemize}
\item Add \texttt{CS1XA3/Project01/project\_analyze.sh}
\item Add \texttt{CS1XA3/Project01/README.md}
\end{itemize}
\subsection{Commit and Push}
\label{sec:orgdacbf36}
\begin{itemize}
\item Add and commit with the following message {\color{purple}EXACTLY} {\color{olive}"Initial Project01 Commit"}
\item Push to GitHub
\end{itemize}
\subsection{Create a Branch}
\label{sec:orgc974042}
\begin{itemize}
\item Create a branch called {\color{purple}project01}
\item Push the branch to github (i.e {\color{purple}git push origin project01})
\item Work from the {\color{purple}project01} branch and {\color{red}only merge with master when your ready to submit Part 1 or Part 2}
\end{itemize}

\newpage

\section{Project Submission Part 1 (Due Feb 2 at 11:59pm)}
\label{sec:orgf82793b}
Merge to the master branch a {\color{purple}WORKING SCRIPT} that:
\begin{itemize}
\item implements at least {\color{purple}20 points} worth of {\color{purple}5 point} features including the
{\color{purple}Script Input} feature
\item documents the features you implement and script usage in the {\color{purple}README.md}
\item documents the {\color{purple}two custom features} you plan to implement (although you
don't need to implement them yet)
\item has a commit/merge message {\color{olive}"Submitting Project 01 Part 1"} {\color{purple}EXACTLY}
\end{itemize}
{\color{red}WARNING} make sure to push to GitHub to complete your submission 

\section{Project Submission Part 2 (Due Feb 16 at 11:59pm)}
\label{sec:orgb48419c}
Merge to the master branch a {\color{purple}WORKING SCRIPT} that:
\begin{itemize}
\item implements at least {\color{purple}20 points} more worth of features
including {\color{purple}at least one 10 point feature}
\item implements {\color{purple}2 custom features} that should be around the same level of
difficulty as the {\color{purple}10 point features}
\item has completed documentation of all features in the {\color{purple}README.md}
\item has a commit/merge message {\color{olive}"Submitting Project 01 Part 2"} {\color{purple}EXACTLY}
\end{itemize}
{\color{red}WARNING} make sure to push to GitHub to complete your submission

\newpage

\section{README}
\label{sec:orged667f8}
You {\color{purple}MUST} document your code in a file \texttt{CS1XA3/Project01/README.md}
\begin{itemize}
\item The document should be styled with {\color{purple}MarkDown} (see
{\color{cyan}\url{https://guides.github.com/features/mastering-markdown/}})
\item The document should describe usage of the script (i.e how to execute, with what arguments, under what conditions)
\item The document should contain a section (header) for each feature (including custom features)
\item You must reference any code used that was found online with a link to the
url (failing to do so will be considered academic dishonesty)
\end{itemize}
An example of the general outline of the document would be as follows (this is
not an exact template you need to follow, you are encouraged to use your best
judgment for constructing a useful README):


\begin{verbatim}
#  CS 1XA3 Project01 - <MyMacId>

## Usage
   Execute this script from project root with:
        chmod +x CS1XA3/Project01/project_analyze.sh
        ./CS1XA3/Project01/project_analyze arg1 arg2 ...
   With possible arguments
        arg1: ....
        arg2: ....
   ....
## Feature 01
 Description: this feature does ....
 Execution: execute this feature by ...
 Reference: some code was taken from [[https://someurl.com]]

## Feature 02
  ...
## Custom Feature SomeFeature
  ...
\end{verbatim}

\newpage

\section{Grading Scheme}
\label{sec:orgfe72720}
\begin{center}
\begin{tabular}{ll}
README Documentation & \textbf{20\%}\\
Custom Features & \textbf{20\%}\\
Other Features & \textbf{60\%}\\
\end{tabular}
\end{center}
{\color{red}WARNING} failure to properly follow instructions (including not cloning
 your repo to the proper directory, not pushing to GitHub, not using the
 correct commit message, etc) will result in {\color{purple}A MARK OF 0}

\subsection{Grading: README Documentation}
\label{sec:org6a8f307}
The README file submission is worth a total of {\color{purple}15 points}
that is projected onto {\color{purple}20\% of the overall grade}, please
refer to the following rubric for details on what criteria you will be marked
upon

\begin{center}
\begin{tabular}{|l|l|l|l|l|}
 & \textbf{Criteria} & \textbf{Points}\\
\hline
\textbf{Style} & Use markdown where appropriate & 2 points\\
 & Readable formatting & 2 points\\
 & Broken up into an appropriate amount of sections & 1 point\\
\hline
\textbf{Correctness} & Instructions for execution are correct & 1 point\\
 & Description of each feature is correct & 4 points\\
\hline
\textbf{Detail} & Instructions for execution have appropriate detail & 1 point\\
 & Description of each feature has appropriate detail & 4 points\\
\end{tabular}
\end{center}

\subsection{Grading: Custom Features}
\label{sec:org43b3325}
You will implement two custom features, each worth {\color{purple}5 points} (so the
two together will be worth a total of {\color{purple}10 points} which
will be projected onto {\color{purple}20\% of the overall grade}), please
refer to the following rubric for details on what criteria each individual
custom feature will be marked upon

\begin{center}
\begin{tabular}{|l|l|l|l|l|}
 & \textbf{Criteria} & \textbf{Points}\\
\hline
\textbf{Creativity} & Is unique, substantially different from other features & 1 point\\
\hline
\textbf{Applicability} & Has an actually useful application & 1 point\\
 & Incorporates skills taught in course & 2 points\\
\hline
\textbf{Correctness} & Works as expected/described in README & 1 point\\
\end{tabular}
\end{center}


\subsection{Criteria: Other Features}
\label{sec:orgf87a718}
You will implement {\color{purple}40 points} of features corresponding
to {\color{purple}60\%} of your overall project mark. Reference the \hyperref[sec:features]{Features}
section for a list of all possible features and their corresponding point
worth. Please refer to the following rubric for details on what criteria each
5 point feature will be marked upon (double each point for features worth 10 points)

\begin{center}
\begin{tabular}{|l|l|l|l|l|}
 & \textbf{Criteria} & \textbf{Points}\\
\hline
\textbf{Correctness} & Works for at least one use case & 1 point\\
 & Accounts for directories/files with special characters & 1 point\\
 & AND/OR not existing AND/OR accounts for command IO failure & \\
 & Is executable as described in README & 1 point\\
 & Works for all use cases & 2 points\\
\end{tabular}
\end{center}

\subsection{Plagiarism / Academic Dishonesty}
\label{sec:orgddc7b5d}
Tools will be used to compare your code with your peers (including previous
years of this course) 
\begin{itemize}
\item Stealing a custom feature idea will be considered plagiarism
\item Using code without referencing the source in your README will be considered
plagiarism.
\item Any account of plagiarism will result in an automatic grade of 0
\end{itemize}

\newpage

\section{Features}
\label{sec:features}
\subsection{Script Input (\textbf{Mandatory}) (5 Points)}
\label{sec:org62b50a9}
\begin{itemize}
\item Make the script interactive (i.e select which feature(s) are executed)
either by providing script arguments or by user prompted input
\item Describe this feature in the {\color{purple}Usage} section of the {\color{purple}README.md} document
rather than in it's own header
\end{itemize}
\subsection{FIXME Log (5 points)}
\label{sec:orga92bccc}
\begin{itemize}
\item Find every file in your repo that has the word {\color{purple}\#FIXME} in the last line
\item Put the list of these file names in  \texttt{CS1XA3/Project01/fixme.log} with each file separated by a newline
\item Create the file \texttt{CS1XA3/Project01/fixme.log} if it doesn't exist, overwrite
it if it does
\end{itemize}
\subsection{Checkout Latest Merge (5 points)}
\label{sec:orgd7b2bcd}
\begin{itemize}
\item Find the most recent commit with the word {\color{purple}merge} (case
insensitive) in the commit message
\item Automatically checkout that commit (so that you're in a detached head
state)
\end{itemize}
\subsection{File Size List (5 points)}
\label{sec:orgef589fb}
\begin{itemize}
\item List all files in the repo (just files not directories) and their sizes in
a human readable format (i.e KB, MB, GB, etc)
\item List the files sorted from largest to smallest
\end{itemize}
\subsection{File Type Count (5 points)}
\label{sec:org03812b2}
\begin{itemize}
\item Using the read command (with a prompt), prompt the user for an extension
(i.e txt, pdf, py, etc)
\item Output the number of files in your repo with that extension
\end{itemize}
\subsection{Find Tag (5 points)}
\label{sec:orge952576}
\begin{itemize}
\item Using the read command (with a prompt, prompt the user for a Tag (any
single word)
\item Create a log file \texttt{CS1XA3/Project01/Tag.log} (where Tag is the name of the
Tag provided) if it doesn't already exist, overwrite it if it does
\item For each python file (i.e ending in {\color{purple}.py}) in the repo, find all lines
that begin with a comment (i.e \texttt{\#}) and include Tag and put them in \texttt{CS1XA3/Project01/Tag.log}
\end{itemize}
\subsection{Switch to Executable (10 points)}
\label{sec:org06f0007}
\begin{itemize}
\item Find all shell scripts (i.e ending in {\color{purple}.sh}) in the repo
\item Create a file \texttt{CS1XA3/Project01/permissions.log} if it doesn't already exist
\item Using the read command, prompt the user to {\color{purple}Change} or
{\color{purple}Restore} (use a prompt that tells the user what to do)
\item If the user selects {\color{purple}Change}:
\begin{itemize}
\item For each shell script, change the permissions so that only people who
have write permissions also have executable permissions (i.e if only
user has write permissions, then only user has executable permissions)
\item Store a log of the file and it's original permissions in
\texttt{CS1XA3/Project01/permissions.log} (overwrite it if it already exists)
\end{itemize}
\item If the user selects {\color{purple}Restore}
\begin{itemize}
\item Restore each file to its original permissions (as specified in
\texttt{CS1XA3/Project01/permissions.log})
\end{itemize}
\end{itemize}
\subsection{Backup and Delete / Restore (10 points)}
\label{sec:orga5ad5e9}
\begin{itemize}
\item Using the read command, prompt the user to {\color{purple}Backup} or
{\color{purple}Restore} (use a prompt that tells the user what to do)
\item If the user selects {\color{purple}Backup}:
\begin{itemize}
\item Create an empty directory \texttt{CS1XA3/Project01/backup} if it doesn't exit
\item Empty the directory \texttt{CS1XA3/Project01/backup} if it does exist
\item Find all files that end in the {\color{purple}.tmp} extension
\begin{itemize}
\item copy them to the \texttt{CS1XA3/Project01/backup} directory
\item delete them from their original location
\item create a file \texttt{CS1XA3/Project01/backup/restore.log} that contains a
list of paths of the files original locations
\end{itemize}
\end{itemize}
\item If the user selects {\color{purple}Restore}:
\begin{itemize}
\item use the file \texttt{CS1XA3/Project01/backup/restore.log} to restore the files
to their original location
\item if the file does not exist, through an error message
\end{itemize}
\end{itemize}
\end{document}